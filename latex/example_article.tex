% The document class plays an importat role.It defines the type of document
% https://tex.stackexchange.com/questions/782/what-are-the-available-documentclass-types-and-their-uses
% My choice would be report, but that has other style for chapter & paragraphs.
\documentclass[a4paper]{article}

\usepackage[utf8]{inputenc}

% image related properties
\usepackage{graphicx}
\graphicspath{ {images/} }

% set page size and margins; + settings headheight for customer header
\usepackage[a4paper, top=3cm, bottom=3cm, left=2cm, right=2cm, marginparwidth=11.75cm, headheight=55pt]{geometry}

% The next three rows define the content for the front page
\author{Martien Huijsmans}
\title{The Latex Example}
\date{\today}

% By default (in most document classes), all paragraphs except for the first paragraph in a section are indented. 
% Next line disables the indentation.
\setlength{\parindent}{0in}

% To add additional spacing between paragraphs, set the "parskip" parameter to an appropriate value. For example: 
\setlength{\parskip}{5pt}

% table: adds \hdashline for horizontal and : for vertical dashed lines
\usepackage{arydshln}

% adds page number to each page
% \pagestyle{myheadings}

% customer headers
%%%%%%%%%%%%%%%%%%%%%%%%%%%%%%%%%%%%%%%%%%%%%%%
% FIRST PAGE ONLY (redefine via \fancypagestyle the plain pagestyle
%%%%%%%%%%%%%%%%%%%%%%%%%%%%%%%%%%%%%%%%%%%%%%%
% source: http://latex.org/forum/viewtopic.php?t=15166
% source: font size https://texblog.org/2012/08/29/changing-the-font-size-in-latex/
\usepackage{fancyhdr}
  \fancypagestyle{plain}{
  \fancyhf{}
  \lhead{
    \begin{tabular}{l}
      Specification
      \\
      Digital Earth Investor Cooperation
    \end{tabular}
  }
  \rhead{
  \begin{tabular}{r}
      Specification
      \\
     Confidenitial
    \end{tabular}
  }
  \lfoot{
    ©  Dragon Quest T.V. 2017  \newline
    \tiny
    All rights are reserved. Reproduction or transmission in whole or in part, in any form or by any means,  \newline
    electronic, mechanical or otherwise, is prohibited without the prior written consent of the copyright owner.
  }
  \rfoot{
    % next line add some white space so that text bottom aligns with bottom of copyright
    \vspace*{0.6\baselineskip}
    \tiny
    Template: RST-0063/Approved
  }
}
%%%%%%%%%%%%%%%%%%%%%%%%%%%%%%%%%%%%%%%%%%%%%%%
% PAGES 2-END 
%%%%%%%%%%%%%%%%%%%%%%%%%%%%%%%%%%%%%%%%%%%%%%%
% source: http://tug.ctan.org/macros/latex/contrib/fancyhdr/fancyhdr.pdf
\pagestyle{fancy}
\lhead{
  \begin{tabular}{l}
  \textbf{Digital Earth Investor} \tabularnewline
  \textbf{Cooperation }
  \\
  \today
  \end{tabular}
}
\chead{
  \begin{tabular}{c}
  Interface Specification \tabularnewline
  Going into Orbit
  \\
  Beta product
  \end{tabular}
}
\rhead{
  \begin{tabular}{r}
  AB-123456-12.01 / Draft
  \\
  \thepage\ of \pageref{LastPage}
  \end{tabular}
}
\lfoot{© Dragon Quest T.V. 2017. All rights reserved.}
\cfoot{}
\rfoot{Confidential}
\renewcommand{\headrulewidth}{0.4pt}
\renewcommand{\footrulewidth}{0.4pt}

% %%%%%%%%%%%%%%%%%%%%%%%%%%%%%%%%%%%%%%%%%%%%%%%%%%%%%%
% %%%%%%%%%%%%%%%%%%%%%%%%%%%%%%%%%%%%%%%%%%%%%%%%%%%%%%
% %%%%%%%%%%%%%%%%%%%%%%%%%%%%%%%%%%%%%%%%%%%%%%%%%%%%%%
% everything before this point is called preamble
\begin{document}

\maketitle
\tableofcontents

\begin{abstract}
Your abstract.
\end{abstract}

\newpage

% starts a level 1 chapter
\section{chapter}

% The [h] is somethimg weard. It keeps image at this place. Else it is at the top of the page.
\begin{figure}[h!]
    \centering
    % fbox puts a box around where the picture is shown
    \fbox{\includegraphics[width=0.3\textwidth]{ys_image}}
    \caption{a nice picture}
    \label{fig:universe1}
\end{figure}
As you can see in the figure \ref{fig:universe1}, it is big.

% starts a level a.b sections
\subsection{Sub-section}

Some text \textbf{in bold} and \textit{in italic} and \underline{underlined}.

% starts a level a.b.c section
%  there is no subsubsubsection
\subsubsection{Sub-section}

\paragraph{A pagrapaph. It is printed in bold}
% Create a bullet list
\begin{itemize}
\item This is the first item
\item This is the second item
\item This is the last itme
\end{itemize}

\subsection{Section with a table}

Tabletext. 
\\
% the | used in next row, tell where the lines shall be
\begin{table}[h!]
\centering
\begin{tabular}{| l || r | r | r | c |}  
\hline
Name&Exam1&Exam2&Exam3&Grade\\
% \hline\hline will add a clear seperation between header of table and content
\hline\hline
John&19& 28&33&C \\  
\hline
Jane&49& 35&60&B  \\
\hline
Jim&76& 38&59&A  \\
\hline
\end{tabular}
\caption{Table to test captions and labels}
\label{table:data1}
\end{table}

Tabletext. 
\\
% the | used in next row, tell where the lines shall be
\begin{table}[h!]
\centering
\begin{tabular}{:c:c:c:c:c:c}  
\hdashline
Name&Exam1&Exam2&Exam3&Grade\\
\hdashline\hdashline
John&19& 28&33&C \\  
\hdashline
Jane&49& 35&60&B  \\
\hdashline
Jim&76& 38&59&A  \\
\hdashline
\end{tabular}
\caption{Table to test captions and labels}
\label{table:data2}
\end{table}

\end{document}