% The document class plays an importat role.It defines the type of document
% https://tex.stackexchange.com/questions/782/what-are-the-available-documentclass-types-and-their-uses
% My choice would be report, but that has other style for chapter & paragraphs.
\documentclass[a4paper]{article}

\usepackage[utf8]{inputenc}

% image related properties
\usepackage{graphicx}
\graphicspath{ {images/} }

% set page size and margins
\usepackage[a4paper, top=2cm, bottom=2cm, left=2cm, right=2cm, marginparwidth=11.75cm]{geometry}

% The next three rows define the content for the front page
\author{Martien Huijsmans}
\title{The Latex Example}
\date{\today}

% By default (in most document classes), all paragraphs except for the first paragraph in a section are indented. 
% Next line disables the indentation.
\setlength{\parindent}{0in}

% To add additional spacing between paragraphs, set the "parskip" parameter to an appropriate value. For example: 
\setlength{\parskip}{5pt}

% adds page number to each page
\pagestyle{myheadings}

% verything before this point is called preamble
\begin{document}

\maketitle

\tableofcontents

\begin{abstract}
Your abstract.
\end{abstract}

\newpage

% starts a level 1 chapter
\section{chapter}

% The [h] is somethimg weard. It keeps image at this place. Else it is at the top of the page.
\begin{figure}[h!]
    \centering
    % fbox puts a box around where the picture is shown
    \fbox{\includegraphics[width=0.3\textwidth]{ys_image}}
    \caption{a nice picture}
    \label{fig:universe1}
\end{figure}
As you can see in the figure \ref{fig:universe1}, it is big.

% starts a level a.b sections
\subsection{Sub-section}

Some text \textbf{in bold} and \textit{in italic} and \underline{underlined}.

% starts a level a.b.c section
%  there is no subsubsubsection
\subsubsection{Sub-section}

\paragraph{A pagrapaph. It is printed in bold}
% Create a bullet list
\begin{itemize}
\item This is the first item
\item This is the second item
\item This is the last itme
\end{itemize}

\subsection{Section with a table}

Tabletext. 
\\
% the | used in next row, tell where the lines shall be
\begin{table}[h!]
\centering
\begin{tabular}{| l || r | r | r | c |}  
\hline
Name&Exam1&Exam2&Exam3&Grade\\
% \hline\hline will add a clear seperation between header of table and content
\hline\hline
John&19& 28&33&C \\  
\hline
Jane&49& 35&60&B  \\
\hline
Jim&76& 38&59&A  \\
\hline
\end{tabular}
\caption{Table to test captions and labels}
\label{table:data}
\end{table}

\end{document}